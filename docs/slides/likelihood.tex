\documentclass{beamer}
\useoutertheme{infolines}
\usepackage[utf8]{inputenc}
\usepackage{default}
%\usepackage{enumitem}
\usepackage{graphicx}
\usepackage[normalem]{ulem}
\newcommand\redout{\bgroup\markoverwith
{\textcolor{red}{\rule[.5ex]{2pt}{2pt}}}\ULon}
\usepackage{xcolor}
\usepackage{textpos}
\usepackage{tikz}   
\usepackage[round]{natbib}
\bibliographystyle{apalike}
\usepackage[T1]{fontenc}
\usepackage{pdfpages}
\usepackage{hyperref}


%\addtobeamertemplate{frametitle}{}{%
%\begin{textblock*}{100mm}(.8\textwidth,-1cm)
%\includegraphics[height=1cm,width=2cm]{opentreelogogrey}
%\end{textblock*}}

\useinnertheme{rectangles}
\usetheme{Copenhagen}
\setbeamertemplate{footline}[frame number]{}
\setbeamertemplate{footline}{}

\title[*]{Phylogenetic inference and likelihood}
\author[*]{Emily Jane McTavish}
\institute[*]{
Life and Environmental Sciences\\
University of California, Merced\\
\texttt{ejmctavish@ucmerced.edu, twitter:snacktavish}\\
}
\date{}
\setbeamertemplate{itemize items}[triangle]

\begin{document}

\begin{frame}
\titlepage
(With thanks to Mark Holder and Paul Lewis for slides) 
\end{frame}




\begin{frame}
Rule: Two taxa that share a character state must be more
closely related to each other than either is to a taxon that
displays a different state.(method suggested by Hennig)\\ 
\textit{Is this a valid rule?}
\end{frame}
%\appendix
%\begin{frame}[allowframebreaks]
% \bibliography{GradPhylo}
%\end{frame}


\setbeamercolor{background canvas}{bg=}
\includepdf[pages=75-82]{../../../markphylo/2017/lec1-IntroStats.pdf}



\begin{frame}
\begin{itemize}
 \item If characters are not polarized (ancestral and descendent states known)
 method can infer unrooted trees.
 \item We can infer tree topology, but be unable to tell paraphyletic from
monophyletic groups.
 \item The outgroup method amounts to inferring an unrooted tree and then
rooting the tree on the branch that leads to an outgroup.
\end{itemize}
\end{frame}


\setbeamercolor{background canvas}{bg=}
\includepdf[pages=4]{../../../markphylo/2017/lec4-ML-CompatPars.pdf}

\begin{frame}
\textbf{problems with this approach}
\begin{itemize}
 \item We don't know polarization
 \item We observe character conflict in real data sets
\end{itemize}
\end{frame}



\setbeamercolor{background canvas}{bg=}
\includepdf[pages=6-23]{../../../markphylo/2017/lec4-ML-CompatPars.pdf}


\begin{frame}
\textbf{Should we expect character conflict?}
\begin{itemize}
 \item Data type?
 \item Evolutionary history?
\end{itemize}
\end{frame}


\begin{frame}
\textbf{How can we deal with character conflict?}
\begin{itemize}
 \item We need to apply an error model
 \item Likelihood provides a measure of surprise under different models
\end{itemize}
\end{frame}


\setbeamercolor{background canvas}{bg=}
\includepdf[pages=15]{../../../Lewis/Likelihood2017.pdf}


\setbeamercolor{background canvas}{bg=}
\includepdf[pages=5]{../../../Lewis/Likelihood2017.pdf}


\setbeamercolor{background canvas}{bg=}
\includepdf[pages=16-22]{../../../Lewis/Likelihood2017.pdf}

\begin{frame}
\textbf{Discussion Question}\\
Is it possible for the EQUAL model to fit a data set better (using the likeli-
hood to measure model fit) than the FLEXIBLE model? Why or why not?
\end{frame}



\begin{frame}
\textbf{Historical aside}\\
\includegraphics[width=0.8\textwidth]{pheneticscladisticstshirt}
\end{frame}

\setbeamercolor{background canvas}{bg=}
\includepdf[pages=20]{../../../Hillis/WoodsHoleMole2017.pdf}




\end{document}
