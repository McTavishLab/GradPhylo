\documentclass{beamer}
\useoutertheme{infolines}
\usepackage[utf8]{inputenc}
\usepackage{default}
%\usepackage{enumitem}
\usepackage{graphicx}
\usepackage[normalem]{ulem}
\newcommand\redout{\bgroup\markoverwith
{\textcolor{red}{\rule[.5ex]{2pt}{2pt}}}\ULon}
\usepackage{xcolor}
\usepackage{textpos}
\usepackage{tikz}   
\usepackage[round]{natbib}
\bibliographystyle{apalike}
\usepackage[T1]{fontenc}
\usepackage{pdfpages}
\usepackage{hyperref}


%\addtobeamertemplate{frametitle}{}{%
%\begin{textblock*}{100mm}(.8\textwidth,-1cm)
%\includegraphics[height=1cm,width=2cm]{opentreelogogrey}
%\end{textblock*}}

\useinnertheme{rectangles}
\usetheme{Copenhagen}
\setbeamertemplate{footline}[frame number]{}
\setbeamertemplate{footline}{}

\title[*]{Tree formats and vocabulary}
\author[*]{Emily Jane McTavish}
\institute[*]{
Life and Environmental Sciences\\
University of California, Merced\\
\texttt{ejmctavish@ucmerced.edu, twitter:snacktavish}\\
}
\date{}
\setbeamertemplate{itemize items}[triangle]

\begin{document}

\begin{frame}
\titlepage
(With thanks to Mark Holder, Paul Lewis, Joe Felsenstein, and David Hillis for slides) 
\end{frame}



\begin{frame}
\begin{center}
 \Large{Phylogenies describe shared ancestry\\
and\\
inform our understanding of evolutionary processes}
\end{center}
\end{frame}


%-----comp meth examples------
\setbeamercolor{background canvas}{bg=}
\includepdf[pages=15-20]{TreTermsBAK.pdf}


\begin{frame}
\begin{centering}
\includegraphics[width=\textwidth]{xkcdherp}
\end{centering}\\
https://xkcd.com/867/
\end{frame}

\setbeamercolor{background canvas}{bg=}
\includepdf[pages=22]{TreTermsBAK.pdf}

\begin{frame}
\begin{centering}
\includegraphics[height=0.8\textheight]{poly}
\end{centering}\\
from wikipedia
\end{frame}


\begin{frame}
more terms:
\begin{itemize}
 \item sister taxa: taxa or monophyletic groups which share a most recent common ancestor
 \item outgroup: taxon that is determined \emph{a priori} to be sister to all other taxa in the analysis. Used for rooting tree
\end{itemize}

\end{frame}

\setbeamercolor{background canvas}{bg=}
\includepdf[pages=25-26]{TreTermsBAK.pdf}

\begin{frame}
Splits\\
\begin{itemize}
 \item It is useful to think of unrooted trees in terms of 'splits'
 \item Each branch in an unrootred tree splits the taxa into two groups.
 \item Membership in those groups can be denoted by ** vs .. 
 \item e.g. a split between 1+2 and 3+4 can be summarized as
 \item 1234
 \item ** ..
\end{itemize}
\end{frame}

\setbeamercolor{background canvas}{bg=}
\includepdf[pages=28]{TreTermsBAK.pdf}

%-----Digression into newick format

\begin{frame}
 \center{\Large a brief digression into newick tree file format}
\end{frame}


\begin{frame}
\begin{centering}
\includegraphics[height=0.4\textheight]{/home/ejmctavish/Desktop/Talks/newicklobster}
\end{centering}\\
 Newick’s Lobster House was the site of an historic 1986 meeting 
 at which a standard was devised for storing descriptions of 
 phylogenetic trees as strings. 
 (Photo from Paul Lewis)
\end{frame}


\setbeamercolor{background canvas}{bg=}
\includepdf[pages=31-43]{TreTermsBAK.pdf}




\begin{frame}
Newick\\
\begin{itemize}
 \item Parenthetical tree format
 \item Rooted vs. unrooted trees are not differentiated
 \item Some programs interpret polytomy at root as `unrooted
 \item Branches and nodes not well differentiated
 \item A name can contain and characters except blanks, colons, semicolons, parentheses, and square brackets
\end{itemize}
\end{frame}

\begin{frame}
Nexus\\
\begin{itemize}
 \item Starts with \#nexus
 \item Can contain blocks of alignments, trees, commands, and more!
 \item Blocks between `begin' and `end'
 \item Trees in Newick format, prepended with [\&U] unrooted or [\&R] rooted
\end{itemize}
\end{frame}

\begin{frame}
Nexus\\
\includegraphics[height=0.3\textheight]{/home/ejmctavish/Desktop/Talks/nexustaxa}\\
\includegraphics[height=0.3\textheight]{/home/ejmctavish/Desktop/Talks/nexusdata}\\
\url{http://hydrodictyon.eeb.uconn.edu/eebedia/index.php/Phylogenetics:_NEXUS_Format}
\end{frame}
\begin{frame}
Nexus\\
\includegraphics[height=0.3\textheight]{/home/ejmctavish/Desktop/Talks/nexustrees}\\
\includegraphics[height=0.2\textheight]{/home/ejmctavish/Desktop/Talks/nexussets}\\
\url{http://hydrodictyon.eeb.uconn.edu/eebedia/index.php/Phylogenetics:_NEXUS_Format}
\end{frame}


\begin{frame}
NeXML\\
\begin{itemize}
 \item Phylogenetic data as XML
 \item Can capture all information from Nexus
 \item Full semantic annotation
 \item Easily extensible
\end{itemize}
\end{frame}

\begin{frame}
NeXML\\
Computer readable, but not very human readable\\
\includegraphics[width=0.75\textwidth]{/home/ejmctavish/Desktop/Talks/nexml}\\
\end{frame}


\begin{frame}
 Phylip (sequence data format)\\
\begin{itemize}
 \item First line must be two integers: <number of taxa> <number of sites>
 \item Sequence ID followed by spaces up to 10 char. 
 \item No duplicate names
 \item Relaxed phylip up to 250 characters followed by a space
\end{itemize}
\includegraphics[width=\textwidth]{/home/ejmctavish/Desktop/Talks/phylip}
\end{frame}

\begin{frame}
Phylip interleaved\\
\includegraphics[height=0.3\textheight]{/home/ejmctavish/Desktop/Talks/phylipinterleaved}\\
Phylip sequential\\
\includegraphics[height=0.3\textheight]{/home/ejmctavish/Desktop/Talks/phylipsequential}
\end{frame}



\begin{frame}
Fasta (sequence data format)\\
\begin{itemize}
 \item Description line before each sequence starts with (">") symbol in the first column
\end{itemize}
\includegraphics[height=0.2\textheight]{/home/ejmctavish/Desktop/Talks/fasta}
\end{frame}



\begin{frame}{An Exercise: Writing and Viewing a Tree by Hand}
    \begin{enumerate}
        \item   Make sure you have:
            \begin{itemize}
                \item   A \textit{good} text editor installed.
                \item   \textit{FigTree} (or some other tree visualizer of choice).
            \end{itemize}
        \item   Create a subdirectory for all our labs, e.g. ''\texttt{\$HOME/projects/GradPhylo}``.
        \item   Create a work subdirectory within this directory for this particular lab, e.g. ''\texttt{lab-01}``.
        \item   Using a command line text editor (nano or or personal favorite) create a simple Newick tree file using a text editor, e.g. ``simple.newick''.
        \item   This tree should reflect the real topological relationships of at least 4 species.
        \item   Visualize it in \textit{FigTree}.
        \item   Export as a NEXUS file.
        \item   Open this second file in a text editor and edit the tip labels.
        \item   Visualize it again.
    \end{enumerate}
\end{frame}

%----------Play Break---------------
\begin{frame}
\frametitle{DIY}
Create a newick tree file in your text editor with the content:
\texttt{(((C,(D,E)),(F,G),A),B);}\\
Save it as 'example.tre'.\\
\begin{itemize}
 \item Draw the tree by hand
 \item Write down all the splits in ..** format.
 \item Load the tree in a tree viewer (e.g. phylo.io, figtree). Re-root  the tree. What rootings make the following true? Which cannot be true?
 \begin{itemize}
 \item A is more closely related to G than it is to C
 \item (C,D,E) is sister to (A,B,F,G)
 \item (C,D) is sister to (A,B,E,F,G)
 \item (C,D,E) is a paraphyletic group
 \item (C,D,E) is a monophyletic group
 \item (A,B,C) is a monophyletic group
 \end{itemize}
\end{itemize}
\end{frame}



\end{document}
