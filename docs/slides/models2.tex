\documentclass{beamer}
\useoutertheme{infolines}
\usepackage[utf8]{inputenc}
\usepackage{default}
%\usepackage{enumitem}
\usepackage{graphicx}
\usepackage[normalem]{ulem}
\newcommand\redout{\bgroup\markoverwith
{\textcolor{red}{\rule[.5ex]{2pt}{2pt}}}\ULon}
\usepackage{xcolor}
\usepackage{textpos}
\usepackage{tikz}   
\usepackage[round]{natbib}
\bibliographystyle{apalike}
\usepackage[T1]{fontenc}
\usepackage{pdfpages}
\usepackage{hyperref}
\usefonttheme[onlymath]{serif}

%\addtobeamertemplate{frametitle}{}{%
%\begin{textblock*}{100mm}(.8\textwidth,-1cm)
%\includegraphics[height=1cm,width=2cm]{opentreelogogrey}
%\end{textblock*}}

\useinnertheme{rectangles}
\usetheme{Copenhagen}
\setbeamertemplate{footline}[frame number]{}
\setbeamertemplate{footline}{}

\title[*]{Phylogenetic likelihood and models}
\author[*]{Emily Jane McTavish}
\institute[*]{
Life and Environmental Sciences\\
University of California, Merced\\
\texttt{ejmctavish@ucmerced.edu, twitter:snacktavish}\\
}
\date{}
\setbeamertemplate{itemize items}[triangle]

\begin{document}

\begin{frame}
\titlepage
(With thanks to Mark Holder and Paul Lewis for slides) 
\end{frame}



\setbeamercolor{background canvas}{bg=}
\includepdf[pages=35]{../../../Lewis/Likelihood2017.pdf}

\setbeamercolor{background canvas}{bg=}
\includepdf[pages=7-8]{../../../markphylo/parsSummaryModels.pdf}


\setbeamercolor{background canvas}{bg=}
\includepdf[pages=36-40]{../../../Lewis/Likelihood2017.pdf}

\setbeamercolor{background canvas}{bg=}
\includepdf[pages=9]{../../../markphylo/parsSummaryModels.pdf}

\begin{frame}
Poisson distibution can be used to explain statistical regularities of rare events\\
$P\left( \textrm{k events in interval} \right) = \frac{{e^{ - \nu } \nu ^k }}{{k!}}$
\begin{itemize}
 \item $\nu$ is the average number of events per interval (rate times time)
 \item $e$ is the number 2.71828... (Euler's number) the base of the natural logarithms
 \item $k$ takes values 0, 1, 2, ...
 \item $k! = k * \left(k - 1\right) * \left(k - 2\right) * … * 2 * 1$ is the factorial of k.
\end{itemize}
\tiny{from wikipedia}
\end{frame}

\begin{frame}
$P\left( \textrm{k events in interval} \right) = \frac{{e^{ - \nu } \nu ^k }}{{k!}}$

$P\left( \textrm{0 events} \right) = \frac{{e^{ - \nu } \nu ^0 }}{{0!}} = e^ {- \nu} = e^{- \mu t}$

$P\left( \textrm{>= 1 events} \right) = 1- e^{- \mu t}$

\end{frame}


\setbeamercolor{background canvas}{bg=}
\includepdf[pages=41-49]{../../../Lewis/Likelihood2017.pdf}



%\setbeamercolor{background canvas}{bg=}
%\includepdf[pages=20-22]{../../../markphylo/parsSummaryModels.pdf}

\setbeamercolor{background canvas}{bg=}
\includepdf[pages=52-75]{../../../Lewis/Likelihood2017.pdf}

\begin{frame}
Iqtree tutorial\end{frame}

%\setbeamercolor{background canvas}{bg=}
%\includepdf[pages=24-33]{../../../markphylo/parsSummaryModels.pdf}








\end{document}
