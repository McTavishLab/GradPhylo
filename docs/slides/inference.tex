\documentclass{beamer}
\useoutertheme{infolines}
\usepackage[utf8]{inputenc}
\usepackage{default}
%\usepackage{enumitem}
\usepackage{graphicx}
\usepackage[normalem]{ulem}
\newcommand\redout{\bgroup\markoverwith
{\textcolor{red}{\rule[.5ex]{2pt}{2pt}}}\ULon}
\usepackage{xcolor}
\usepackage{textpos}
\usepackage{tikz}   
\usepackage[round]{natbib}
\bibliographystyle{apalike}
\usepackage[T1]{fontenc}
\usepackage{pdfpages}
\usepackage{hyperref}


%\addtobeamertemplate{frametitle}{}{%
%\begin{textblock*}{100mm}(.8\textwidth,-1cm)
%\includegraphics[height=1cm,width=2cm]{opentreelogogrey}
%\end{textblock*}}

\useinnertheme{rectangles}
\usetheme{Copenhagen}
\setbeamertemplate{footline}[frame number]{}
\setbeamertemplate{footline}{}

\title[*]{Phylogenetic inference from data}
\author[*]{Emily Jane McTavish}
\institute[*]{
Life and Environmental Sciences\\
University of California, Merced\\
\texttt{ejmctavish@ucmerced.edu, twitter:snacktavish}\\
}
\date{}
\setbeamertemplate{itemize items}[triangle]

\begin{document}

\begin{frame}
\titlepage
(With thanks to Jeet Sukumaran and Mark Holder for slides) 
\end{frame}




\begin{frame}
\Large
How do we figure out what tree captures the relationships we're interested in?
\end{frame}

\begin{frame}{Phylogenetic Inference}
    \begin{itemize}[<+->]
        \item   We cannot see, measure, collect, or otherwise obtain a phylogeny directly from nature.
        \item   We have to \textit{infer} it from data that we \textit{can} collect from nature.
        \item   A variety of different inferential approaches and data are used, with both increasing in sophistication and complexity over time.
        \item   Fundamental perspectives in all these approaches:
            \begin{itemize}
                \item   Current patterns of biodiversity has been generated by processes of: (1) speciation, (2) extinction, and (3) character modification.
                \item   The phylogeny is an abstract representation (``model'') of this diversification process.
            \end{itemize}
    \end{itemize}
\end{frame}



\begin{frame}
\Large{Enormous numbers of topologies to consider}
\includegraphics[scale=0.4]{/home/ejmctavish/Desktop/Talks/numberoftrees}
\end{frame}

\begin{frame}
\Large{Enormous numbers of topologies to consider}
\includegraphics[scale=0.4]{/home/ejmctavish/Desktop/Talks/numberoftreesastro}
\end{frame}

\begin{frame}
Estimating a tree from character data\\
Tree construction:
\begin{itemize}
 \item strictly algorithmic approaches - use a “recipe” to construct a tree
  \item optimality based approaches - choose a way to “score” a trees and then search for the tree that has the best score.
\end{itemize}

\end{frame}



\setbeamercolor{background canvas}{bg=}
\includepdf[pages=48-56]{../../../markphylo/2017/lec1-IntroStats.pdf}


\setbeamercolor{background canvas}{bg=}
\includepdf[pages=59-62]{../../../markphylo/2017/lec1-IntroStats.pdf}


\begin{frame}
Rule: Two taxa that share a character state must be more
closely related to each other than either is to a taxon that
displays a different state.(method suggested by Hennig)\\ 
\textit{Is this a valid rule?}
\end{frame}
%\appendix
%\begin{frame}[allowframebreaks]
% \bibliography{GradPhylo}
%\end{frame}


\setbeamercolor{background canvas}{bg=}
\includepdf[pages=75-81]{../../../markphylo/2017/lec1-IntroStats.pdf}


\begin{frame}
Breakout room! What are the relationships between these taxa?
\end{frame}
%\appendix

\setbeamercolor{background canvas}{bg=}
\includepdf[pages=82]{../../../markphylo/2017/lec1-IntroStats.pdf}




\begin{frame}
If characters are not polarized (ancestral and descendent states known)
this method can infer unrooted trees. \\
\medskip
\end{frame}



\setbeamercolor{background canvas}{bg=}
\includepdf[pages=6-8]{../../../markphylo/2017/lec4-ML-CompatPars.pdf}

\begin{frame}
\textbf{Character conflict}\\
Two characters are compatible if they can both be mapped on the same
tree so that all of the character states displayed could be homologous.\\
\pause
\medskip
Incompatible characters are evidence of homoplasy in the data
\pause
\medskip
Homoplasy literally means the “same change” has occurred more than once
in the evolutionary history of the group.
The presence of homoplasy undermines Hennigian and Parsimony analyses.
\end{frame}


\includepdf[pages=19-21]{../../../markphylo/2017/lec4-ML-CompatPars.pdf}

\includepdf[pages=23]{../../../markphylo/2017/lec4-ML-CompatPars.pdf}

\begin{frame}
In this class we will focus on Maximum Likelihood and Bayesian statistical estimates for evolutionary models.
\end{frame}



\end{document}
