\documentclass{beamer}
\useoutertheme{infolines}
\usepackage[utf8]{inputenc}
\usepackage{default}
%\usepackage{enumitem}
\usepackage{graphicx}
\usepackage[normalem]{ulem}
\newcommand\redout{\bgroup\markoverwith
{\textcolor{red}{\rule[.5ex]{2pt}{2pt}}}\ULon}
\usepackage{xcolor}
\usepackage{textpos}
\usepackage{tikz}   
\usepackage[round]{natbib}
\bibliographystyle{apalike}
\usepackage[T1]{fontenc}
\usepackage{pdfpages}
\usepackage{hyperref}


%\addtobeamertemplate{frametitle}{}{%
%\begin{textblock*}{100mm}(.8\textwidth,-1cm)
%\includegraphics[height=1cm,width=2cm]{opentreelogogrey}
%\end{textblock*}}

\useinnertheme{rectangles}
\usetheme{Copenhagen}
\setbeamertemplate{footline}[frame number]{}
\setbeamertemplate{footline}{}

\title[*]{Phylogenetic thinking}
\author[*]{Emily Jane McTavish}
\institute[*]{
Life and Environmental Sciences\\
University of California, Merced\\
\texttt{ejmctavish@ucmerced.edu, twitter:snacktavish}\\
}
\date{}
\setbeamertemplate{itemize items}[triangle]

\begin{document}

\begin{frame}
\titlepage
(With thanks to Mark Holder, Paul Lewis, Joe Felsenstein, and David Hillis for slides) 
\end{frame}



\begin{frame}
\begin{center}
 \Large{Phylogenies describe shared ancestry\\
and\\
inform our understanding of evolutionary processes}
\end{center}
\end{frame}


%-----comp meth examples------
\setbeamercolor{background canvas}{bg=}
\includepdf[pages=3-14]{TreTermsBAK.pdf}

\begin{frame}
Epidemiology\\
 \includegraphics[width=\textwidth]{/home/ejmctavish/Desktop/Talks/nextstrain_covid}
 \url{https://nextstrain.org/ncov/gisaid/global} (deep dive next week)
\end{frame}


%-----------------
\begin{frame}
Phylogenies can reveal suprising patterns
 \includegraphics[width=0.75\textwidth]{butterflycut}
\end{frame}

\begin{frame}
Phylogenies can reveal suprising patterns
 \includegraphics[width=\textwidth]{butterfly}
 \citep{joron_chromosomal_2011}
\end{frame}



\begin{frame}
What evolutionary processes can drive these patterns?
\begin{itemize}
 \item Convergence
 \item Horizontal gene transfer
 \item Incomplete lineage sorting
 \item ?
\end{itemize}
\pause
We will discuss how to recognize and (try to) differentiate these processes.
\end{frame}



\begin{frame}
\textit{The challenge of phylogenetics:}\\
The tree is real. There is some true past history that happened!\\
\medskip
\pause
How do we find out what that history was?
\end{frame}


\begin{frame}
Estimating a tree from character data\\
Tree construction:
\begin{itemize}
 \item strictly algorithmic approaches - use a “recipe” to construct a tree
  \item optimality based approaches - choose a way to “score” a trees and then search for the tree that has the best score.
\end{itemize}

Expressing support for aspects of the tree:
\begin{itemize}
 \item  bootstrapping,
 \item testing competing trees against each other,
 \item posterior probabilities (in Bayesian approaches).
\end{itemize}
\end{frame}

\begin{frame}
Different data can drive different conclusions
 \includegraphics[width=\textwidth]{batspptree}\\
Species relationships between echolocating and nonecholocating bats (after Teeling 2009).
Left: inferences from DNA sequence data.\\
Right: traditional species relationships inferred from morphological characters (and limited sequence data).
\citep{hahn_irrational_2016}
\end{frame}


\appendix
\begin{frame}[allowframebreaks]
 \bibliography{GradPhylo}
\end{frame}



\end{document}
